\documentclass[11pt,a4paper]{article}

% --- PAQUETES ---
\usepackage[utf8]{inputenc}
\usepackage[spanish]{babel}
\usepackage{graphicx}
\usepackage{booktabs}
\usepackage{geometry}
\usepackage{float}
\usepackage{xcolor}
\usepackage{titlesec}
\usepackage{setspace}
\usepackage{hyperref} % Para que el link de GitHub sea clickable
\geometry{margin=2.5cm}

% --- DISEÑO DE TÍTULOS ---
\titleformat{\section}{\large\bfseries\color{black}}{}{0em}{}[\titlerule]

\begin{document}

% --- PORTADA ---
\begin{titlepage}
    \centering
    \vspace*{2cm}
    {\huge\bfseries Aprendizaje Automático II\par}
    \vspace{1cm}
    {\Large\bfseries Práctica 1: Clasificación de Imágenes mediante Redes Neuronales Convolucionales (CNN)\par}
    \vspace{2.5cm}
    
    {\large\textit{Autoras:}\par}
    \vspace{0.5cm}
    {\Large Laura Aguiar \par}
    {\Large Lucía Hernández \par}
    
    \vspace{2cm}
    {\large\textit{Repositorio del Proyecto:}\par}
    \vspace{0.3cm}
    \texttt{\url{https://github.com/Lauuaguiar/AA2_Pr-ctica_1}}
    
    \vfill
    {\large \today \par}
    \vspace{1cm}
    \rule{\linewidth}{0.5mm}
\end{titlepage}

\newpage

\section{Introducción y Objetivo}
Este informe detalla el desarrollo de un sistema de \textbf{Aprendizaje Profundo} diseñado para clasificar imágenes en seis categorías: edificios, bosques, glaciares, montañas, mar y calles. El objetivo central fue aplicar una metodología científica de experimentación para optimizar una CNN hasta superar el \textbf{85\% de precisión}.

\section{Arquitectura del Sistema}
El modelo se basa en una arquitectura de Red Neuronal Convolucional, que procesa la información en tres etapas fundamentales:
\begin{itemize}
    \item \textbf{Capas Convolucionales:} Actúan como extractores de características visuales (bordes, texturas y formas complejas).
    \item \textbf{Dropout (0.5):} Técnica de regularización que "apaga" neuronas aleatoriamente durante el entrenamiento. Esto evita que la red dependa en exceso de píxeles específicos, reduciendo el sobreajuste.
    \item \textbf{Early Stopping:} Mecanismo que monitoriza el error en el conjunto de prueba y detiene el proceso si no hay mejora en 5 épocas, asegurando la mejor versión del modelo.
\end{itemize}



\section{Metodología de Experimentación}
Se han ejecutado cinco configuraciones, ajustando parámetros de forma aislada para comprender su impacto en el rendimiento global.

\begin{table}[H]
\centering
\caption{Tabla comparativa de configuraciones y resultados}
\begin{spacing}{1.2}
\begin{tabular}{@{}llccccc@{}}
\toprule
\textbf{Exp.} & \textbf{Estrategia} & \textbf{LR} & \textbf{Batch} & \textbf{Capas} & \textbf{Data Aug.} & \textbf{Accuracy} \\ \midrule
Config 1 & Base & 0.001 & 32 & 2 & No & 80\% \\
Config 2 & Suavizado LR & 0.0001 & 32 & 2 & No & 82\% \\
Config 3 & Aumento Batch & 0.0001 & 64 & 2 & No & 76\% \\
Config 4 & Capacidad Visual & 0.0001 & 64 & 3 & No & 83\% \\
Config 5 & \textbf{Robustez Final} & 0.0001 & 64 & 3 & \textbf{Sí} & \textbf{86\%} \\ \bottomrule
\end{tabular}
\end{spacing}
\end{table}

\section{Análisis de la Configuración Ganadora (Config 5)}
La Configuración 5 alcanzó el \textbf{86\% de precisión}, cumpliendo con los objetivos de la práctica.

\subsection{El impacto del Data Augmentation}
La diferencia clave radica en el \textbf{Aumento de Datos}. Al introducir variaciones aleatorias como rotaciones y zooms, el modelo se ve obligado a aprender conceptos visuales abstractos en lugar de memorizar imágenes. Esto explica por qué, con la misma arquitectura que la Config 4, se logró subir un 3\% la precisión.

\begin{figure}[H]
    \centering
    \includegraphics[width=0.8\textwidth]{config5/config5_plots.png}
    \caption{Curvas de Precisión y Pérdida (Config 5). La estabilidad de las curvas valida el uso de un Learning Rate bajo (0.0001).}
\end{figure}

\subsection{Diagnóstico mediante Matriz de Confusión}
Al analizar la matriz generada, determinamos:
\begin{itemize}
    \item \textbf{Fortalezas:} Excelente identificación de paisajes de tipo \textit{forest} (98\% de recall).
    \item \textbf{Confusiones Comunes:} Ligero solapamiento entre \textit{buildings} y \textit{street}, debido a la similitud de materiales geométricos (asfalto y concreto).
\end{itemize}

\begin{figure}[H]
    \centering
    \includegraphics[width=0.65\textwidth]{config5/config5_confusion_matrix.png}
    \caption{Matriz de confusión final del mejor modelo.}
\end{figure}

\section{Conclusiones}
La experimentación demuestra que la calidad del entrenamiento es tan relevante como la arquitectura. El uso de un \textbf{Learning Rate bajo}, una \textbf{arquitectura de 3 capas} y la \textbf{regularización mediante aumento de datos}, han sido los pilares fundamentales para obtener un sistema robusto y capaz de generalizar con éxito.

\end{document}